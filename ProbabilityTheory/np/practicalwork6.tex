\documentclass[12pt]{article}
\usepackage[utf8]{inputenc}
\usepackage[russian]{babel}
\usepackage{amsmath}
\usepackage{amsfonts}
\usepackage{amssymb}
\usepackage{graphicx}
\usepackage{geometry}
\usepackage{hyperref}
\usepackage{float}
\usepackage{booktabs}
\usepackage{array}
\geometry{a4paper, margin=2.5cm}

\begin{document}

\begin{titlepage}
\thispagestyle{empty}
\begin{center}
    \textbf{Университет ИТМО}\\
    \vspace{1em}
    Факультет Программной Инженерии и Компьютерной Техники\\
    \vspace{1em}
    {\Large Теория вероятностей}\\
    \vspace{15em}
    {\Large Лабораторная работа № 6}\\[1em]
    Вариант: \textbf{15}
\end{center}
\vspace{12em}
\begin{flushright}
    Выполнил:\\
    Чжун Цзяцзюнь\\
    Группа: P3210\\
    Преподаватель:\\
    Селина Елена Георгиевна\\
\end{flushright}
\vspace{2em}
\vspace{\fill}
\begin{center}
    Санкт-Петербург\\
    2025 г.
\end{center}
\end{titlepage}

\tableofcontents
\newpage

\section{Постановка задачи}

Дана таблица распределения 100 предприятий по производственным средствам $X$ (тыс. ден. ед.) и по суточной выработке $Y$ (тыс. единиц продукции). Известно, что между $X$ и $Y$ существует линейная корреляционная зависимость. 

\textbf{Требуется:}
\begin{enumerate}
    \item[а)] найти уравнение прямой регрессии $y$ на $x$;
    \item[б)] построить уравнение эмпирической линии регрессии и случайные точки выборки $(X, Y)$.
\end{enumerate}

\section{Исходные данные}

Таблица распределения для варианта 15:

\begin{table}[H]
\centering
\begin{tabular}{|c|c|c|c|c|c|c|c|c||c|}
\hline
\backslashbox{$X$}{$Y$} & 1200 & 2700 & 4200 & 6700 & 8200 & 9700 & 11200 & 12700 & $m_x$ \\
\hline
20 & 4 & 2 & 5 & --- & --- & --- & --- & --- & 11 \\
\hline
520 & --- & --- & 7 & 5 & 2 & --- & --- & --- & 14 \\
\hline
1020 & --- & --- & --- & 9 & 14 & 6 & --- & --- & 29 \\
\hline
1520 & --- & --- & --- & 7 & 8 & 6 & --- & --- & 21 \\
\hline
2020 & --- & --- & --- & --- & 4 & 5 & 7 & --- & 16 \\
\hline
2520 & --- & --- & --- & --- & --- & 3 & 2 & 4 & 9 \\
\hline
\hline
$m_y$ & 4 & 2 & 12 & 21 & 28 & 20 & 9 & 4 & 100 \\
\hline
\end{tabular}
\caption{Исходная таблица распределения}
\end{table}

Где:
\begin{itemize}
    \item $X$ — производственные средства (тыс. ден. ед.)
    \item $Y$ — суточная выработка (тыс. единиц продукции)
    \item $m_x$ — частоты по строкам
    \item $m_y$ — частоты по столбцам
    \item $n = 100$ — общее число наблюдений
\end{itemize}

\section{Решение}

\subsection{Расчетная таблица}

Для подсчета числовых характеристик (выборочных средних $\bar{x}$ и $\bar{y}$, выборочных средних квадратичных отклонений $s_x$ и $s_y$ и выборочного корреляционного момента $s_{xy}$) составляем расчетную таблицу.

\begin{table}[H]
\centering
\scalebox{0.85}{%
\small
\begin{tabular}{|c|c|c|c|c|c|c|c|c||c|c|c|c|}
\hline
$i$ & $x_i$ & $m_{x_i}$ & $m_{x_i}x_i$ & $x_i^2m_{x_i}$ & $\sum m_{ij}y_j$ & $y_1$ & $y_2$ & ... & $y_8$ & $\sum m_{ij}x_iy_j$ \\
\hline
1 & 20 & 11 & 220 & 4400 & 4(1200)+2(2700)+5(4200)=30600 & 4 & 2 & 5 & --- & 4(20)(1200)+... \\
\hline
2 & 520 & 14 & 7280 & 3785600 & 7(4200)+5(6700)+2(8200)=79800 & --- & --- & 7 & --- & ... \\
\hline
3 & 1020 & 29 & 29580 & 30171600 & 9(6700)+14(8200)+6(9700)=233100 & --- & --- & --- & --- & ... \\
\hline
4 & 1520 & 21 & 31920 & 48518400 & 7(6700)+8(8200)+6(9700)=170700 & --- & --- & --- & --- & ... \\
\hline
5 & 2020 & 16 & 32320 & 65286400 & 4(8200)+5(9700)+7(11200)=164000 & --- & --- & --- & --- & ... \\
\hline
6 & 2520 & 9 & 22680 & 57153600 & 3(9700)+2(11200)+4(12700)=393900 & --- & --- & --- & 4 & ... \\
\hline
\hline
$\sum$ & --- & 100 & 124000 & 204919600 & 1072100 & --- & --- & --- & --- & ... \\
\hline
\end{tabular}%
}
\end{table}

Вычисления для столбца $\sum m_{ij}y_j$ по каждой строке:
\begin{align*}
i=1: & \quad 4 \cdot 1200 + 2 \cdot 2700 + 5 \cdot 4200 = 4800 + 5400 + 21000 = 31200 \\
i=2: & \quad 7 \cdot 4200 + 5 \cdot 6700 + 2 \cdot 8200 = 29400 + 33500 + 16400 = 79300 \\
i=3: & \quad 9 \cdot 6700 + 14 \cdot 8200 + 6 \cdot 9700 = 60300 + 114800 + 58200 = 233300 \\
i=4: & \quad 7 \cdot 6700 + 8 \cdot 8200 + 6 \cdot 9700 = 46900 + 65600 + 58200 = 170700 \\
i=5: & \quad 4 \cdot 8200 + 5 \cdot 9700 + 7 \cdot 11200 = 32800 + 48500 + 78400 = 159700 \\
i=6: & \quad 3 \cdot 9700 + 2 \cdot 11200 + 4 \cdot 12700 = 29100 + 22400 + 50800 = 102300 \\
\text{Сумма:} & \quad 31200 + 79300 + 233300 + 170700 + 159700 + 102300 = 776500
\end{align*}

Проверка по столбцам:
\begin{align*}
\sum m_y y_j &= 4 \cdot 1200 + 2 \cdot 2700 + 12 \cdot 4200 + 21 \cdot 6700 + 28 \cdot 8200 \\
&\quad + 20 \cdot 9700 + 9 \cdot 11200 + 4 \cdot 12700 \\
&= 4800 + 5400 + 50400 + 140700 + 229600 + 194000 + 100800 + 50800 \\
&= 776500
\end{align*}

\subsection{Вычисление выборочных средних}

Выборочное среднее $\bar{x}$:
\[
\bar{x} = \frac{\sum m_{x_i} x_i}{n} = \frac{20 \cdot 11 + 520 \cdot 14 + 1020 \cdot 29 + 1520 \cdot 21 + 2020 \cdot 16 + 2520 \cdot 9}{100}
\]
\[
\bar{x} = \frac{220 + 7280 + 29580 + 31920 + 32320 + 22680}{100} = \frac{124000}{100} = 1240 \text{ тыс. ден. ед.}
\]

Выборочное среднее $\bar{y}$:
\begin{align*}
\bar{y} &= \frac{\sum m_{y_j} y_j}{n} \\
&= \frac{1200 \cdot 4 + 2700 \cdot 2 + 4200 \cdot 12 + 6700 \cdot 21 + 8200 \cdot 28 + 9700 \cdot 20 + 11200 \cdot 9 + 12700 \cdot 4}{100} \\
&= \frac{4800 + 5400 + 50400 + 140700 + 229600 + 194000 + 100800 + 50800}{100} \\
&= \frac{776500}{100} = 7765 \text{ тыс. ед.}
\end{align*}

\subsection{Вычисление выборочных дисперсий}

Для вычисления выборочной дисперсии $s_x^2$ нужно найти $\sum m_{x_i} x_i^2$:
\begin{align*}
\sum m_{x_i} x_i^2 &= 11 \cdot 20^2 + 14 \cdot 520^2 + 29 \cdot 1020^2 + 21 \cdot 1520^2 + 16 \cdot 2020^2 + 9 \cdot 2520^2 \\
&= 11 \cdot 400 + 14 \cdot 270400 + 29 \cdot 1040400 + 21 \cdot 2310400 \\
&\quad + 16 \cdot 4080400 + 9 \cdot 6350400 \\
&= 4400 + 3785600 + 30171600 + 48518400 + 65286400 + 57153600 \\
&= 204920000
\end{align*}

Выборочная дисперсия $s_x^2$:
\[
s_x^2 = \frac{1}{n-1} \left( \sum m_{x_i} x_i^2 - \frac{1}{n} \left( \sum m_{x_i} x_i \right)^2 \right)
\]
\[
s_x^2 = \frac{1}{99} \left( 204920000 - \frac{124000^2}{100} \right) = \frac{1}{99} \left( 204920000 - 153760000 \right)
\]
\[
s_x^2 = \frac{51160000}{99} = 516767.68
\]
\[
s_x = \sqrt{516767.68} \approx 718.87 \text{ тыс. ден. ед.}
\]

Для вычисления $s_y^2$ нужно найти $\sum m_{y_j} y_j^2$:
\begin{align*}
\sum m_{y_j} y_j^2 &= 4 \cdot 1200^2 + 2 \cdot 2700^2 + 12 \cdot 4200^2 + 21 \cdot 6700^2 \\
&\quad + 28 \cdot 8200^2 + 20 \cdot 9700^2 + 9 \cdot 11200^2 + 4 \cdot 12700^2 \\
&= 4 \cdot 1440000 + 2 \cdot 7290000 + 12 \cdot 17640000 + 21 \cdot 44890000 \\
&\quad + 28 \cdot 67240000 + 20 \cdot 94090000 + 9 \cdot 125440000 + 4 \cdot 161290000 \\
&= 5760000 + 14580000 + 211680000 + 942690000 \\
&\quad + 1882720000 + 1881800000 + 1128960000 + 645160000 \\
&= 6713350000
\end{align*}

Выборочная дисперсия $s_y^2$:
\[
s_y^2 = \frac{1}{99} \left( 6713350000 - \frac{776500^2}{100} \right) = \frac{1}{99} \left( 6713350000 - 602952250 \right)
\]
\[
s_y^2 = \frac{683827500}{99} = 6907348.49
\]
\[
s_y = \sqrt{6907348.49} \approx 2628.18 \text{ тыс. ед.}
\]

\subsection{Вычисление корреляционного момента}

Для вычисления $s_{xy}$ нужно найти $\sum \sum m_{ij} x_i y_j$. Вычисляем для каждой ячейки с ненулевой частотой:

\begin{align*}
\sum \sum m_{ij} x_i y_j &= 4(20)(1200) + 2(20)(2700) + 5(20)(4200) \\
&\quad + 7(520)(4200) + 5(520)(6700) + 2(520)(8200) \\
&\quad + 9(1020)(6700) + 14(1020)(8200) + 6(1020)(9700) \\
&\quad + 7(1520)(6700) + 8(1520)(8200) + 6(1520)(9700) \\
&\quad + 4(2020)(8200) + 5(2020)(9700) + 7(2020)(11200) \\
&\quad + 3(2520)(9700) + 2(2520)(11200) + 4(2520)(12700)
\end{align*}

Вычисляем по строкам:
\begin{align*}
\text{Строка 1:} & \quad 96000 + 108000 + 420000 = 624000 \\
\text{Строка 2:} & \quad 15288000 + 17420000 + 8528000 = 41236000 \\
\text{Строка 3:} & \quad 61506000 + 117096000 + 59418000 = 238020000 \\
\text{Строка 4:} & \quad 71176000 + 99712000 + 88584000 = 259472000 \\
\text{Строка 5:} & \quad 66256000 + 97970000 + 158368000 = 322594000 \\
\text{Строка 6:} & \quad 73332000 + 56448000 + 128016000 = 257796000 \\
\text{Сумма:} & \quad 1119742000
\end{align*}

Корреляционный момент:
\[
s_{xy} = \frac{1}{n-1} \left( \sum \sum m_{ij} x_i y_j - \frac{1}{n} \left( \sum m_{x_i} x_i \right) \left( \sum m_{y_j} y_j \right) \right)
\]
\[
s_{xy} = \frac{1}{99} \left( 1119742000 - \frac{124000 \cdot 776500}{100} \right)
\]
\[
s_{xy} = \frac{1}{99} \left( 1119742000 - 962860000 \right) = \frac{156882000}{99} = 1584040.40
\]

\subsection{Коэффициент корреляции}

\[
r_{xy} = \frac{s_{xy}}{s_x \cdot s_y} = \frac{1584040.40}{718.87 \cdot 2628.18} = \frac{1584040.40}{1889332.89} \approx 0.8384
\]

\subsection{Уравнение регрессии}

Уравнение эмпирической линии регрессии $y$ на $x$ имеет вид:
\[
y - \bar{y} = r_{xy} \frac{s_y}{s_x} (x - \bar{x})
\]

Подставляем значения:
\[
y - 7765 = 0.8384 \cdot \frac{2628.18}{718.87} (x - 1240)
\]
\[
y - 7765 = 0.8384 \cdot 3.6558 (x - 1240)
\]
\[
y - 7765 = 3.065 (x - 1240)
\]
\[
y = 3.065x - 3800.6 + 7765
\]
\[
\boxed{y = 3.065x + 3964.4}
\]

Или в округленном виде:
\[
\boxed{y \approx 3.07x + 3964}
\]

\section{Графическое представление}
\begin{figure}[H]
    \centering
    \includegraphics[width=0.95\textwidth]{regression_plot.png}
    \label{fig:mapping}
\end{figure}

\section{Выводы}

\begin{enumerate}
    \item Получено уравнение регрессии: $y = 3.07x + 3964$
    \item Коэффициент корреляции $r_{xy} \approx 0.8384$ указывает на сильную положительную линейную связь между производственными средствами и суточной выработкой
    \item С увеличением производственных средств на 1 тыс. ден. ед. суточная выработка в среднем увеличивается примерно на 3.07 тыс. единиц продукции
    \item Высокое значение коэффициента корреляции (близкое к 1) свидетельствует о том, что линейная модель хорошо описывает зависимость между переменными
\end{enumerate}

\end{document}