\documentclass[12pt]{article}
\usepackage[utf8]{inputenc}
\usepackage[russian]{babel}
\usepackage{amsmath}
\usepackage{amsfonts}
\usepackage{amssymb}
\usepackage{graphicx}
\usepackage{geometry}
\usepackage{hyperref}
\usepackage{float}
\geometry{a4paper, margin=2.5cm}
\begin{document}
\begin{titlepage}
\thispagestyle{empty}
\begin{center}
    \textbf{Университет ИТМО}\\
    \vspace{1em}
    Факультет Программной Инженерии и Компьютерной Техники\\
    \vspace{1em}
    {\Large ТФКП}\\
    \vspace{15em}
    {\Large Лабораторная работа \textnumero 2}\\[1em]
    {\Large \textbf{Построение конформных отображений}}\\[1em]
    Вариант:\textbf{16}
\end{center}
\vspace{12em}
\begin{flushright}
    Выполнил:\\
    Чжун Цзяцзюнь\\
    Группа: P3210\\
    Преподаватель:\\
    Богачёв Владимир Александрович\\
\end{flushright}
\vspace{2em}
\vspace{\fill}
\begin{center}
    Санкт-Петербург 2025 г.
\end{center}
\end{titlepage}
\tableofcontents
\newpage
\section{Задание}
Для варианта 16 необходимо построить конформное отображение между:
\begin{itemize}
    \item \textbf{Рисунок 5}: Угловая область 
    \item \textbf{Рисунок 6}: Полоса на комплексной плоскости
   
    \centering
    \includegraphics[width=0.95\textwidth]{pic.png}
\end{itemize}
\section{Аналитическое описание множеств}
\subsection{Рисунок 5}
Исходная область представляет собой угловой сектор:
\begin{equation}
    G_1 = \left\{ z \in \mathbb{C} : \frac{\pi}{4} < \arg(z) < \frac{3\pi}{4}, \, |z| > 0 \right\}
\end{equation}
Это сектор комплексной плоскости с углом раствора $\frac{\pi}{2}$ (от $\frac{\pi}{4}$ до $\frac{3\pi}{4}$), симметричный относительно положительной мнимой оси. В полярных координатах:
\begin{equation}
    z = r e^{i\theta}, \quad \text{где } r > 0, \quad \frac{\pi}{4} < \theta < \frac{3\pi}{4}
\end{equation}
\subsection{Рисунок 6}
Целевая область представляет собой полуполосу:
\begin{equation}
    G_2 = \left\{ w \in \mathbb{C} : \text{Re}(w) > 0, \, 0 < \text{Im}(w) < \pi \right\}
\end{equation}
Это горизонтальная полуполоса с вершиной в начале координат, ограниченная:
\begin{itemize}
    \item Справа: уходит в бесконечность ($\text{Re}(w) \to +\infty$)
    \item Снизу: действительной полуосью ($\text{Im}(w) = 0$, $\text{Re}(w) > 0$)
    \item Сверху: прямой $\text{Im}(w) = \pi$, $\text{Re}(w) > 0$
    \item Слева: мнимой полуосью от $0$ до $\pi i$ (не включая границу)
\end{itemize}
\section{Построение конформного отображения}
\subsection{Прямое отображение: $G_1 \to G_2$}
Для построения конформного отображения из $G_1$ в $G_2$ используем следующую композицию классических преобразований (соответствует комбинации из таблицы: №4 для возведения в степень, №16 для логарифма, с поворотами через умножение на $e^{i\theta}$).
\subsubsection{Поворот на $-90^\circ$}
Применяем преобразование:
\begin{equation}
    w_0 = -i z = e^{-i \pi / 2} z
\end{equation}
\textbf{Действие:}
\begin{itemize}
    \item Угол $\arg(z)$ смещается на $-\pi/2$: если $\frac{\pi}{4} < \arg(z) < \frac{3\pi}{4}$, то $-\frac{\pi}{4} < \arg(w_0) < \frac{\pi}{4}$
    \item Сектор становится симметричным относительно положительной вещественной оси
\end{itemize}
Получаем область:
\begin{equation}
    G_0 = \left\{ w_0 \in \mathbb{C} : -\frac{\pi}{4} < \arg(w_0) < \frac{\pi}{4}, \, |w_0| > 0 \right\}
\end{equation}
\subsubsection{Возведение в квадрат}
Применяем преобразование:
\begin{equation}
    w_1 = w_0^2
\end{equation}
\textbf{Действие:}
\begin{itemize}
    \item Угол $\arg(w_0)$ удваивается: $-\frac{\pi}{2} < \arg(w_1) < \frac{\pi}{2}$
    \item Радиус возводится в квадрат: $|w_1| = |w_0|^2$
    \item Сектор с углом $\frac{\pi}{2}$ преобразуется в правую полуплоскость ($\text{Re}(w_1) > 0$)
\end{itemize}
Получаем область:
\begin{equation}
    G_1' = \left\{ w_1 \in \mathbb{C} : \text{Re}(w_1) > 0 \right\}
\end{equation}
\subsubsection{Поворот на $90^\circ$}
Применяем преобразование:
\begin{equation}
    w_2 = i w_1 = e^{i \pi / 2} w_1
\end{equation}
\textbf{Действие:}
\begin{itemize}
    \item Правая полуплоскость поворачивается на $90^\circ$ и становится верхней полуплоскостью ($\text{Im}(w_2) > 0$)
    \item Аргумент лежит в $(0, \pi)$
\end{itemize}
Получаем область:
\begin{equation}
    G_2' = \left\{ w_2 \in \mathbb{C} : \text{Im}(w_2) > 0 \right\}
\end{equation}
\subsubsection{Логарифмическое отображение}
Применяем преобразование:
\begin{equation}
    w = \ln w_2
\end{equation}
где $\ln$ --- главная ветвь логарифма с $\arg \in (0, \pi)$.
\textbf{Действие:}
\begin{itemize}
    \item Верхняя полуплоскость отображается в горизонтальную полосу $0 < \text{Im}(w) < \pi$
    \item Действительная часть: $\text{Re}(w) = \ln |w_2|$, которая может быть от $-\infty$ до $+\infty$
    \item Для получения $\text{Re}(w) > 0$ можно ограничить исходные точки $|z| > 1$, но поскольку области бесконечны, полуполоса является частью полной полосы
\end{itemize}
\subsection{Итоговое прямое отображение}
Композиция всех преобразований:
\begin{equation}
    f(z) = \ln \left( i (-i z)^2 \right) = \ln (-i z^2) = \ln(-i) + 2 \ln z = -i \frac{\pi}{2} + 2 \ln z
\end{equation}
(с учётом ветви логарифма для $\arg \in (0, \pi)$).
\subsection{Обратное отображение: $G_2 \to G_1$}
Обратное отображение строится путём обращения каждого шага в обратном порядке с соответствующими ветвями:
\begin{enumerate}
    \item Обратный логарифм: $v = e^w$
    \item Обратный поворот (деление на $i$): $u = v / i = -i v$
    \item Обратное возведение в квадрат (квадратный корень): $s = \sqrt{u}$, ветвь с $\text{Re}(s) > 0$, $\arg(s) \in (-\pi/2, \pi/2)$
    \item Обратный поворот (умножение на $i$): $z = i s$
\end{enumerate}
Итоговое обратное отображение:
\begin{equation}
    f^{-1}(w) = i \sqrt{-i \, e^{w}}
\end{equation}
где квадратный корень берётся с ветвью $\text{Re} > 0$, а экспонента согласуется по ветвям.
\section{Программная реализация}
Программа реализована на языке Python с использованием библиотек \texttt{numpy} и \texttt{matplotlib}.
\subsection{Основные этапы программы}
\begin{enumerate}
    \item \textbf{Генерация точек исходной области}: создаётся сетка точек в секторе $\frac{\pi}{4} < \arg(z) < \frac{3\pi}{4}$ в полярных координатах (радиус от 0.1 до 3 для визуализации)
    \item \textbf{Применение преобразований}: последовательно применяются все промежуточные отображения (поворот, квадрат, поворот, логарифм)
    \item \textbf{Визуализация}: отрисовка исходной области, промежуточных результатов и финальной области с фильтрацией на $\text{Re}(w) > 0$
    \item \textbf{Визуализация сетки}: построение конформной сетки (радиальные линии и окружности) для наглядности сохранения углов
\end{enumerate}

\section{Результаты}
\subsection{Визуализация преобразований}
На рисунках ниже представлены результаты применения конформных отображений:
\begin{figure}[H]
    \centering
    \includegraphics[width=0.95\textwidth]{conformal_mapping_variant16.png}
    \caption{Пошаговое конформное отображение области из Рисунка 5 в Рисунок 6}
    \label{fig:mapping}
\end{figure}
\begin{figure}[H]
    \centering
    \includegraphics[width=0.95\textwidth]{conformal_grid_variant16.png}
    \caption{Трансформация конформной сетки при отображениях}
    \label{fig:grid}
\end{figure}
\subsection{Анализ результатов}
\begin{itemize}
    \item Поворот $-i z$ симметрирует сектор относительно вещественной оси
    \item Преобразование $w = z^2$ удваивает угол, преобразуя сектор с углом $\frac{\pi}{2}$ в правую полуплоскость
    \item Поворот $i w$ переводит в верхнюю полуплоскость
    \item Логарифм $\ln w$ даёт горизонтальную полосу $0 < \text{Im}(w) < \pi$
    \item Конформная сетка показывает сохранение углов между кривыми при преобразованиях
    \item Радиальные линии исходной области преобразуются в горизонтальные линии в полосе, а угловые --- в вертикальные
    \item Область корректно отображается из углового сектора в горизонтальную полуполосу (с $\text{Re}(w) > 0$ для $|z| > 1$)
\end{itemize}

\section{Выводы}
В ходе выполнения лабораторной работы:
\begin{enumerate}
    \item Построено аналитическое описание заданных областей (Рисунок 5 и Рисунок 6)
    \item Составлено конформное отображение из исходной области в целевую с использованием композиции классических преобразований
    \item Получено обратное отображение
    \item Реализована программа на Python для визуализации преобразований
    \item Продемонстрировано сохранение конформной структуры при отображениях
\end{enumerate}
\end{document}