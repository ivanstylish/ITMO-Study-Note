\documentclass{article}
\usepackage[utf8]{inputenc}
\usepackage[russian]{babel}
\usepackage{amsmath}
\usepackage{amssymb}
\usepackage{geometry}
\geometry{a4paper, margin=1in}

\begin{document}

\section*{Весна '25. Предварительная волна. Вариант 1}

\subsection*{1.1 Приведите пример линейной формы в пространстве полиномов}
Приведите пример линейной формы в пространстве многочленов
\[f(p) = p(a) \quad a \in \mathbb{R}\]

\subsection*{1.2 Как найти матрицу билинейной формы в некотором базисе}
\[\{e_i\}^n_1 - \text{базис. Тогда матрица билин. формы имеет вид } \beta_{ij} = b(e_i, e_j)\]

\subsection*{1.3 В чём заключается смысл немого суммирования}
Если в выражении встречается один и тот же индекс в верхнем и нижнем положении, то по нему подразумевается суммирование от 1 до n (где n — размерность пространства).

\subsection*{1.4 Как выглядит закон преобразования тензора типа (1, 1)}
\[T^i_j\] — компоненты тензора типа (1,1) в старом базисе,
\[T'^m_l\] — компоненты этого же тензора в новом базисе,
\[P = (P^i_k)\] — матрица перехода от старого базиса к новому,
\[Q = P^{-1}\] — матрица перехода от нового базиса к старому,
\[T'^m_l = P^m_i T^i_j Q^j_l\]

\subsection*{1.5 Напишите закон преобразования матрицы оператора при смене базиса}
Пусть \(\varphi \in \text{Hom}_K(V, W)\), а в пространствах заданы базисы:
\[V : \{e_i\}^n_{i=1}\]
\[W : \{g_k\}^m_{k=1}\]

Причем известно, что \(T = \{t_{ij}\}\) — матрица перехода из базиса \(\{e\}\) в базис \(\{e'\}\), а матрица \(S = \{s_{kl}\}\) — матрица перехода из базиса \(\{g\}\) в базис \(\{g'\}\).
Матрица оператора при замене базисов преобразуется как \(A'_\varphi = S^{-1} A T\)

\subsection*{2.1 Как связаны размерности ядра и образа оператора}
\[\dim_K \ker \varphi + \dim_K \operatorname{Im} \varphi = \dim_K V\]

\subsection*{2.2 Найти собственные значения линейного оператора, матрица которого}
\[\det\left(\begin{pmatrix}
1-\lambda & 2 \\
2 & 1-\lambda
\end{pmatrix}\right) = (1-\lambda)^2 - 2^2 = 0\]
\[(1-\lambda-2)(1-\lambda+2) = 0\]
\[(-\lambda-1)(3-\lambda) = 0\]
\[\lambda = -1 \quad \lambda = 3\]

\subsection*{2.3 Сформулируйте спектральную теорему для диагонализируемого оператора}
Если линейный оператор A на конечномерном векторном пространстве диагонализируем, то существует такой базис пространства, в котором матрица оператора A является диагональной, и её диагональные элементы — это собственные значения оператора.

\subsection*{2.4 Определите алгебраические и геометрические кратности собственных чисел оператора, если в жордановом базисе его матрица имеет вид}
\[\left(\begin{array}{ccc}
1 & 1 & 0 \\
0 & 1 & 1 \\
0 & 0 & 1
\end{array}\right)\]
Собственное число 1, алгебраическая кратность 3, геометрическая кратность 1.

\subsection*{2.5 Запишите матрицу полинома p(x) от диагональной матрицы A}
\[p(A) = \left(\begin{array}{cccc}
p(\lambda_1) & 0 & \cdots & 0 \\
0 & p(\lambda_2) & \cdots & 0 \\
\vdots & \vdots & \ddots & \vdots \\
0 & 0 & \cdots & p(\lambda_n)
\end{array}\right)\]

\subsection*{3.1 Каким образом из евклидова пространства можно получить нормированное}
Евклидово пространство — частный случай нормированного пространства. Чтобы получить нормированное пространство из евклидова, достаточно использовать норму
\[\|x\| = \sqrt{\langle x, x\rangle}\]

\subsection*{3.2 Вычислите скалярное произведение векторов \(x = (1, 2)^T\) и \(y = (0, 3)^T\) в базисе, матрица Грама которого}
\[G = \left(\begin{array}{cc}
2 & 1 \\
1 & 1
\end{array}\right)\]
\[\langle x, y \rangle = x^T G y = (1, 2) \begin{pmatrix}2 & 1\\1 & 1\end{pmatrix} \begin{pmatrix}0\\3\end{pmatrix} = (1, 2) \begin{pmatrix}3\\3\end{pmatrix} = 1\cdot3 + 2\cdot3 = 9\]

\subsection*{3.3 Как найти коэффициенты Фурье вектора в ортонормированном базисе}
В ортонормированном базисе координаты вектора (коэффициенты Фурье) находятся с помощью скалярного произведения. Если базис \(\{e_1, e_2, ..., e_n\}\) ортонормированный, то для любого вектора \(v\) коэффициенты высчитываются как:
\[c_i = \langle v, e_i \rangle\]

\subsection*{3.4 Что такое сигнатура квадратичной формы}
Сигнатура квадратичной формы - это набор из двух чисел \((p, q)\), где p - число положительных собственных значений (или положительных квадратов в нормальном виде), q - число отрицательных собственных значений (или отрицательных квадратов в нормальном виде).

\subsection*{3.5 Сформулируйте свойства унитарного оператора в комплексном евклидовом пространстве}
1. Изометрия:
\[\langle \psi(x), \psi(y) \rangle = \langle x, y \rangle\]
2. Сохранение нормы:
\[\|\psi(x)\| = \|x\|\]
3. Свойство сопряженного:
\[\psi^* = \psi^{-1}\]

\newpage

\section*{Весна '25. Предварительная волна. Вариант 2}

\subsection*{1.1 Напишите определение линейной формы}
Линейной формой на пространстве V называется такая функция \( f : V \rightarrow K \), что \(\forall v_1, v_2 \in V\), \(\forall \lambda \in K\) выполняется:
1. Аддитивность: \( f(v_1 + v_2) = f(v_1) + f(v_2) \)
2. Однородность: \( f(\lambda v) = \lambda f(v) \)

\subsection*{1.2 Пусть билинейная форма задана своей матрицей \( \begin{pmatrix} 4 & 1 \\ 2 & 3 \end{pmatrix} \) в некотором базисе. Представьте её в виде суммы симметричной и антисимметричной компонент}
\[B = \begin{pmatrix} 4 & 1 \\ 2 & 3 \end{pmatrix}\]
\[B_S = \frac{1}{2}(B + B^T) = \frac{1}{2} \left( \begin{pmatrix} 4 & 1 \\ 2 & 3 \end{pmatrix} + \begin{pmatrix} 4 & 2 \\ 1 & 3 \end{pmatrix} \right) = \frac{1}{2} \begin{pmatrix} 8 & 3 \\ 3 & 6 \end{pmatrix} = \begin{pmatrix} 4 & 1.5 \\ 1.5 & 3 \end{pmatrix}\]
\[B_{AS} = \frac{1}{2}(B - B^T) = \frac{1}{2} \left( \begin{pmatrix} 4 & 1 \\ 2 & 3 \end{pmatrix} - \begin{pmatrix} 4 & 2 \\ 1 & 3 \end{pmatrix} \right) = \frac{1}{2} \begin{pmatrix} 0 & -1 \\ 1 & 0 \end{pmatrix} = \begin{pmatrix} 0 & -0.5 \\ 0.5 & 0 \end{pmatrix}\]
\[B = B_S + B_{AS} = \begin{pmatrix} 4 & 1.5 \\ 1.5 & 3 \end{pmatrix} + \begin{pmatrix} 0 & -0.5 \\ 0.5 & 0 \end{pmatrix} = \begin{pmatrix} 4 & 1 \\ 2 & 3 \end{pmatrix}\]

\subsection*{1.3 Что является тензором линейной формы}
Тензором линейной формы является ковариантный тензор ранга 1 (тензор типа (0,1)) — элемент сопряжённого пространства \( V^* \).

\subsection*{1.4 Как может быть найден определитель квадратной матрицы с помощью символа Леви-Чевита}
Пусть матрица A имеет размерность n, тогда
\[\det A = \varepsilon_{i_1 i_2 ... i_n} a_{1 i_1} a_{2 i_2} ... a_{n i_n}\]
(или аналогичная формула с суммированием по всем индексам).

\subsection*{1.5 Матрица линейного оператора \( \varphi \) в базисе \( e_1, e_2 \) некоторого линейного пространства является матрица \( \begin{pmatrix} -3 & 1 \\ 2 & -1 \end{pmatrix} \). Найдите матрицу линейного оператора в базисе \( e_1' = e_2, e_2' = e_1 + e_2 \)}
Матрица перехода \( T = \begin{pmatrix} 0 & 1 \\ 1 & 1 \end{pmatrix} \), её обратная \( T^{-1} = \begin{pmatrix} -1 & 1 \\ 1 & 0 \end{pmatrix} \).
Новая матрица оператора: \( A' = T^{-1} A T = \begin{pmatrix} -1 & 1 \\ 1 & 0 \end{pmatrix} \begin{pmatrix} -3 & 1 \\ 2 & -1 \end{pmatrix} \begin{pmatrix} 0 & 1 \\ 1 & 1 \end{pmatrix} = \begin{pmatrix} 5 & -2 \\ -3 & 1 \end{pmatrix} \begin{pmatrix} 0 & 1 \\ 1 & 1 \end{pmatrix} = \begin{pmatrix} -2 & 3 \\ 1 & -2 \end{pmatrix} \).

\subsection*{2.1 Что такое ядро линейного оператора?}
Ядро линейного оператора \( A : V \rightarrow W \) — это множество всех векторов из пространства \( V \), которые оператор \( A \) переводит в нулевой вектор пространства \( W \)
\[\ker A = \{v \in V | A(v) = 0_W\}\]

\subsection*{2.2 Сформулируйте определение собственного вектора и собственного значения оператора \( A \)}
Ненулевой вектор \( x \in V \) называется собственным вектором оператора \( A \), если \( A(x) = \lambda x \). Число \( \lambda \in K \) называется при этом собственным значением оператора \( A \), отвечающим собственному вектору \( x \).

\subsection*{2.3 Сформулируйте критерии диагонализируемости оператора \( A \)}
1. Оператор \( A \) диагонализируем тогда и только тогда, когда для каждого его собственного значения \( \lambda \) алгебраическая кратность равна геометрической кратности.
2. Сумма размерностей собственных подпространств равна размерности пространства V (характеристический многочлен раскладывается на линейные множители над полем K, и для каждого собственного значения выполняется условие равенства кратностей).

\subsection*{2.4 Определите алгебраические и геометрические кратности собственных чисел оператора, если в жордановом базисе его матрица имеет вид \( \begin{pmatrix} 0 & 1 & 0 \\ 0 & 0 & 1 \\ 0 & 0 & 0 \end{pmatrix} \)}
Собственное число \( \lambda = 0 \), алгебраическая кратность 3, геометрическая кратность 1.

\subsection*{2.5 Запишите матрицу полинома \( p(x) \) от диагональной матрицы \( A \)}
\[p(A) = \begin{pmatrix}
p(\lambda_1) & 0 & 0 \\
0 & p(\lambda_2) & 0 \\
0 & 0 & p(\lambda_3)
\end{pmatrix}\]

\subsection*{3.1 Какое пространство называется комплексным евклидовым пространством?}
Линейное пространство \( X \) над \( \mathbb{C} \) называется комплексным евклидовым (унитарным) пространством, если на нём задана эрмитова форма \( \langle x, y \rangle \) (скалярное произведение), удовлетворяющая свойствам:
1. \( \langle x, y \rangle = \overline{\langle y, x \rangle} \)
2. \( \langle \lambda x_1 + \mu x_2, y \rangle = \lambda \langle x_1, y \rangle + \mu \langle x_2, y \rangle \)
3. \( \langle x, x \rangle > 0 \) для \( x \neq 0 \)

\subsection*{3.2 Приведите пример скалярного произведения в пространстве квадратных матриц}
\[\langle A, B \rangle = \text{tr}(A^* B)\]
где \( A^* \) — эрмитово сопряжённая матрица.

\subsection*{3.3 Как найти ортогональный проектор на подпространство, если задан ортонормированный базис}
\[P_L(x) = \sum_{i=1}^{k} \langle x, e_i \rangle e_i\]
где \( \{e_1, ..., e_k\} \) — ортонормированный базис подпространства \( L \).

\subsection*{3.4 Запишите нормальный вид квадратичной формы, если её сигнатура \( (p, q) = (2, 3) \)}
\[Q(x) = x_1^2 + x_2^2 - x_3^2 - x_4^2 - x_5^2\] (или любой другой вид с 2 знаками "+" и 3 знаками "-").

\subsection*{3.5 Каким свойством обладает матрица эрмитова оператора в ортонормированном базисе}
Если оператор \( T \) является эрмитовым (самосопряжённым), то в любом ортонормированном базисе его матрица \( A \) удовлетворяет:
\[A = A^*\]
где \( A^* \) — эрмитово сопряжённая матрица.

\newpage

\section*{Весна ‘25. Предварительная волна. Вариант 3}

\subsection*{1.1 Приведите пример линейной формы в пространстве геометрических векторов}
\[f(\vec{v}) = \langle \vec{a}, \vec{v} \rangle \quad \text{где } \vec{a} - \text{фиксированный вектор}\]

\subsection*{1.2 Как найти антисимметричную компоненту билинейной формы}
\[b^{AS}(x, y) = \frac{1}{2} (b(x, y) - b(y, x))\]

\subsection*{1.3 Какой валентностью обладает полилинейная форма валентности \((p, q)\) после операции свёртки}
\((p-1, q-1)\)

\subsection*{1.4 Дайте определение символа Леви-Чевита}
\[\varepsilon_{i_1 i_2 ... i_n} =
\begin{cases}
+1 & \text{если } (i_1, i_2, ..., i_n) - \text{чётная перестановка чисел } 1,2,...,n \\
-1 & \text{если } (i_1, i_2, ..., i_n) - \text{нечётная перестановка чисел } 1,2,...,n \\
0 & \text{иначе (есть повторяющиеся индексы)}
\end{cases}\]

\subsection*{1.5 Напишите определение матрицы линейного оператора A в базисе \(\{e_1, e_2, ..., e_n\}\)}
Матрицей линейного оператора A в этом базисе называется квадратная матрица \(A=(a_{ij})\) размера n×n, элементы которой определяются следующим образом:
\[A(e_j) = \sum_{i=1}^n a_{ij} e_i \quad \text{для } j = 1, 2, ..., n\]

\subsection*{2.1 Что такое ядро линейного оператора}
Ядро линейного оператора \(A: V \to W\) — это множество всех векторов из пространства V, которые оператор \(A\) переводит в нулевой вектор пространства W
\[\ker A = \{v \in V | A(v) = 0_W\}\]

\subsection*{2.2 Найти собственные значения линейного оператора, матрица которого}
\[\det \left( \begin{pmatrix}
1-\lambda & 2 \\
2 & 1-\lambda
\end{pmatrix} \right) = (1-\lambda)^2 - 2^2 = 0\]
\[(1-\lambda - 2)(1-\lambda + 2) = 0\]
\[(-\lambda - 1)(3-\lambda) = 0\]
\[\lambda = -1 \quad \lambda = 3\]

\subsection*{2.3 Сформулируйте критерии диагонализируемости оператора A}
1. Оператор A диагонализируем тогда и только тогда, когда для каждого его собственного значения λ алгебраическая кратность равна геометрической кратности.
2. Сумма размерностей собственных подпространств равна размерности пространства V (характеристический многочлен раскладывается на линейные множители над полем K, и для каждого собственного значения выполняется условие равенства кратностей).

\subsection*{2.4 Сформулируйте основную теорему о структуре нильпотентного оператора}
Пусть N — нильпотентный оператор на конечномерном пространстве V. Тогда пространство V раскладывается в прямую сумму циклических относительно N подпространств. Каждое такое подпространство порождено цепочкой векторов вида \(v, N(v), N^2(v), ..., N^{k-1}(v)\), где \(N^k(v) = 0\). Размерность пространства равна сумме размерностей этих циклических подпространств. Количество таких подпространств равно размерности ядра оператора N.

\subsection*{2.5 Запишите матрицу полинома \(p(x)\) от диагональной матрицы A}
\[p(A) =
\begin{pmatrix}
p(\lambda_1) & 0 & \cdots & 0 \\
0 & p(\lambda_2) & \cdots & 0 \\
\vdots & \vdots & \ddots & \vdots \\
0 & 0 & \cdots & p(\lambda_n)
\end{pmatrix}\]

\subsection*{3.1 Сформулируйте определение метрического тензора}
Пусть g — билинейная симметричная форма (скалярное произведение). Тогда совокупность чисел \(g_{ij} = g(e_i, e_j)\), вычисленных в базисе \(\{e_1, ..., e_n\}\), называется метрическим тензором.

\subsection*{3.2 Пусть \(x_1\) и \(x_2\) - ортогональные векторы. При каких α и β выполняется равенство \(αx_1 = βx_2\)}
Равенство \(αx_1 = βx_2\) выполняется только при \(α = β = 0\), так как ненулевые ортогональные векторы линейно независимы.

\subsection*{3.3 Какое подпространство называют ортогональным дополнением}
Ортогональным дополнением подпространства L линейного пространства X со скалярным произведением называется множество:
\[L^\perp = \{x \in X : \langle x, y \rangle = 0 \quad \forall y \in L\}\]

\subsection*{3.4 Какому необходимому и достаточному условию должны удовлетворять главные миноры отрицательно определённой квадратичной формы}
Главные миноры \(D_k\) (определители матриц формы в базисе, составленные из первых k строк и столбцов) отрицательно определённой квадратичной формы должны удовлетворять условию:
\[(-1)^k D_k > 0 \quad \text{для всех } k = 1, 2, ..., n\]

\subsection*{3.5 Сформулируйте определение унитарного оператора}
Линейный оператор \( \psi \) в унитарном (комплексном евклидовом) пространстве называется унитарным, если он сохраняет скалярное произведение:
\[\langle \psi(x), \psi(y) \rangle = \langle x, y \rangle \quad \forall x, y \in X\]
Эквивалентные определения:
1. Сохранение нормы: \(\| \psi(x) \| = \| x \|\)
2. Обратный оператор совпадает с сопряжённым: \( \psi^{-1} = \psi^* \)

\newpage

\section*{Весна '25. Предварительная волна. Вариант 4}

\subsection*{1.1 Напишите определение линейной формы}
Линейной формой на пространстве V называется такая функция \( f : V \rightarrow K \), что \(\forall v_1, v_2 \in V\), \(\forall \lambda \in K\) выполняется:
1. Аддитивность: \( f(v_1 + v_2) = f(v_1) + f(v_2) \)
2. Однородность: \( f(\lambda v) = \lambda f(v) \)

\subsection*{1.2 Пусть билинейная форма задана своей матрицей \( \begin{pmatrix} 2 & 3 \\ -1 & 0 \end{pmatrix} \) в некотором базисе. Представьте её в виде суммы симметричной и антисимметричной формы}
\[B = \begin{pmatrix} 2 & 3 \\ -1 & 0 \end{pmatrix}\]
\[B_S = \frac{1}{2}(B + B^T) = \frac{1}{2} \left( \begin{pmatrix} 2 & 3 \\ -1 & 0 \end{pmatrix} + \begin{pmatrix} 2 & -1 \\ 3 & 0 \end{pmatrix} \right) = \frac{1}{2} \begin{pmatrix} 4 & 2 \\ 2 & 0 \end{pmatrix} = \begin{pmatrix} 2 & 1 \\ 1 & 0 \end{pmatrix}\]
\[B_{AS} = \frac{1}{2}(B - B^T) = \frac{1}{2} \left( \begin{pmatrix} 2 & 3 \\ -1 & 0 \end{pmatrix} - \begin{pmatrix} 2 & -1 \\ 3 & 0 \end{pmatrix} \right) = \frac{1}{2} \begin{pmatrix} 0 & 4 \\ -4 & 0 \end{pmatrix} = \begin{pmatrix} 0 & 2 \\ -2 & 0 \end{pmatrix}\]
\[B = B_S + B_{AS} = \begin{pmatrix} 2 & 1 \\ 1 & 0 \end{pmatrix} + \begin{pmatrix} 0 & 2 \\ -2 & 0 \end{pmatrix} = \begin{pmatrix} 2 & 3 \\ -1 & 0 \end{pmatrix}\]

\subsection*{1.3 В чём заключается смысл немого суммирования}
Если в выражении встречается один и тот же индекс в верхнем и нижнем положении, то по нему подразумевается суммирование от 1 до n (где n — размерность пространства).

\subsection*{1.4 Как выглядит закон преобразования тензора типа (2, 0)}
\(T^{ij}\) — компоненты тензора T типа (2,0) в старом базисе,
\(T'^{kl}\) — компоненты этого же тензора в новом базисе,
\(P = (P^i_k)\) — матрица перехода от старого базиса к новому (новые базисные векторы выражаются через старые: \(e'_k = P^i_k e_i\)),
\[T'^{kl} = P^k_i P^l_j T^{ij}\]
(Суммирование по i и j).

\subsection*{1.5 Матрицей линейного оператора \( \varphi \) в базисе \( e_1, e_2 \) некоторого линейного пространства является матрица \( \begin{pmatrix} 1 & 4 \\ -3 & 0 \end{pmatrix} \). Найдите матрицу линейного оператора в базисе \( e_1' = 2e_1, e_2' = e_2 \)}
Матрица перехода \( T = \begin{pmatrix} 2 & 0 \\ 0 & 1 \end{pmatrix} \), её обратная \( T^{-1} = \begin{pmatrix} 1/2 & 0 \\ 0 & 1 \end{pmatrix} \).
Новая матрица оператора: \( A' = T^{-1} A T = \begin{pmatrix} 1/2 & 0 \\ 0 & 1 \end{pmatrix} \begin{pmatrix} 1 & 4 \\ -3 & 0 \end{pmatrix} \begin{pmatrix} 2 & 0 \\ 0 & 1 \end{pmatrix} = \begin{pmatrix} 0.5 & 2 \\ -3 & 0 \end{pmatrix} \begin{pmatrix} 2 & 0 \\ 0 & 1 \end{pmatrix} = \begin{pmatrix} 1 & 2 \\ -6 & 0 \end{pmatrix} \).

\subsection*{2.1 Какую размерность имеет образ оператора \( \varphi \), определённого в \( \mathbb{R}^4 \), если размерность ядра равна 2}
По теореме о раноге и дефекте: \(\dim \operatorname{Im} \varphi = \dim \mathbb{R}^4 - \dim \ker \varphi = 4 - 2 = 2\).

\subsection*{2.2 Сформулируйте определение собственного вектора и собственного значения оператора A.}
Ненулевой вектор \( x \in V \) называется собственным вектором оператора \( \varphi \), если \( \varphi(x) = \lambda x \). Число \( \lambda \in K \) называется при этом собственным значением оператора \( \varphi \), отвечающим собственному вектору \( x \).

\subsection*{2.3 Сформулируйте спектральную теорему для диагонализируемого оператора}
Если линейный оператор A на конечномерном векторном пространстве диагонализируем, то существует такой базис пространства, в котором матрица оператора A является диагональной, и её диагональные элементы — это собственные значения оператора.

\subsection*{2.4 Определите алгебраические и геометрические кратности собственных чисел оператора если в жордановом базисе его матрица имеет вид \( \begin{pmatrix} 0 & 1 & 0 \\ 0 & 0 & 0 \\ 0 & 0 & 1 \end{pmatrix} \)}
Собственное число 0: алгебраическая кратность 2, геометрическая кратность 1 (один жорданов блок размера 2).
Собственное число 1: алгебраическая кратность 1, геометрическая кратность 1.

\subsection*{2.5 Найдите \( e^A \) если \( A = \begin{pmatrix} 1 & 0 & 0 \\ 0 & 0 & 1 \\ 0 & 0 & 0 \end{pmatrix} \)}
Матрица A блочно-диагональная. \( A = \begin{pmatrix} 1 & 0 & 0 \\ 0 & B & 0 \\ 0 & 0 & 0 \end{pmatrix} \), где \( B = \begin{pmatrix} 0 & 1 \\ 0 & 0 \end{pmatrix} \) — нильпотентная матрица 2-го порядка (\(B^2=0\)).
Тогда \( e^A = \begin{pmatrix} e^1 & 0 & 0 \\ 0 & e^B & 0 \\ 0 & 0 & e^0 \end{pmatrix} = \begin{pmatrix} e & 0 & 0 \\ 0 & I + B + \frac{B^2}{2!} + ... & 0 \\ 0 & 0 & 1 \end{pmatrix} = \begin{pmatrix} e & 0 & 0 \\ 0 & \begin{pmatrix} 1 & 1 \\ 0 & 1 \end{pmatrix} & 0 \\ 0 & 0 & 1 \end{pmatrix} = \begin{pmatrix} e & 0 & 0 \\ 0 & 1 & 1 \\ 0 & 0 & 1 \\ 0 & 0 & 0 \end{pmatrix} \). Исправляем размерность: \( e^A = \begin{pmatrix} e & 0 & 0 \\ 0 & 1 & 1 \\ 0 & 0 & 1 \end{pmatrix} \).

\subsection*{3.1 Сформулируйте определение метрического тензора}
Пусть g — билинейная симметричная форма (скалярное произведение). Тогда совокупность чисел \(g_{ij} = g(e_i, e_j)\), вычисленных в базисе \(\{e_1, ..., e_n\}\), называется метрическим тензором.

\subsection*{3.2 Приведите пример скалярного произведения в пространстве квадратных матриц}
\[\langle A, B \rangle = \operatorname{tr}(A^T B)\]

\subsection*{3.3 Какое подпространство называют ортогональным дополнением}
Ортогональным дополнением подпространства L линейного пространства X со скалярным произведением называется множество:
\[L^\perp = \{x \in X : \langle x, y \rangle = 0 \quad \forall y \in L\}\]

\subsection*{3.4 Запишите квадратичную форму по её матрице \( \begin{pmatrix} 1 & 0 & -1 \\ 0 & 1 & 2 \\ -1 & 2 & 0 \end{pmatrix} \)}
\[Q(x) = x^T A x = (x_1, x_2, x_3) \begin{pmatrix} 1 & 0 & -1 \\ 0 & 1 & 2 \\ -1 & 2 & 0 \end{pmatrix} \begin{pmatrix} x_1 \\ x_2 \\ x_3 \end{pmatrix} = x_1^2 + x_2^2 + 0\cdot x_3^2 + 2\cdot(0)x_1x_2 + 2\cdot(-1)x_1x_3 + 2\cdot(2)x_2x_3\]
\[Q(x) = x_1^2 + x_2^2 - 2x_1x_3 + 4x_2x_3\]

\subsection*{3.5 Сформулируйте свойства спектра ортогонального оператора в вещественном евклидовом пространстве}
1. Все собственные значения по модулю равны 1 (\( |\lambda| = 1 \)).
2. Собственные подпространства, отвечающие разным собственным значениям, ортогональны.
3. Оператор может не быть диагонализируемым над \( \mathbb{R} \) (комплексные собственные значения появляются парами \( e^{\pm i\phi} \)), но пространство раскладывается в ортогональную сумму инвариантных подпространств размерности 1 и 2.

\newpage

\section*{Весна '25. Предварительная волна. Вариант 5}

\subsection*{1.1 Что из себя представляют элементы сопряжённого пространства}
Элементы сопряжённого пространства \( V^* \) — это линейные функционалы (линейные формы) на V, то есть линейные отображения \( f : V \rightarrow \mathbb{F} \), где \( \mathbb{F} \) — поле, над которым определено векторное пространство V.

\subsection*{1.2 Дайте определение билинейной формы на линейном пространстве \( V \)}
Билинейной формой на линейном пространстве V называется функция \( b : V \times V \rightarrow \mathbb{K} \), линейная по каждому аргументу:
1. \( b(\lambda_1 x_1 + \lambda_2 x_2, y) = \lambda_1 b(x_1, y) + \lambda_2 b(x_2, y) \)
2. \( b(x, \mu_1 y_1 + \mu_2 y_2) = \mu_1 b(x, y_1) + \mu_2 b(x, y_2) \)
для всех \( x, x_1, x_2, y, y_1, y_2 \in V \), \( \lambda_1, \lambda_2, \mu_1, \mu_2 \in \mathbb{K} \).

\subsection*{1.3 Сколько различных тензоров можно образовать с помощью свёртки тензора типа (2,2)}
Тензор типа (2,2) имеет 4 индекса. Свертка возможна по разным парам индексов (верхний-нижний). Можно свернуть первый верхний с первым нижним, первый верхний со вторым нижним, второй верхний с первым нижним, второй верхный со вторым нижним. Таким образом, можно образовать 4 различных тензора типа (1,1) путём свёртки.

\subsection*{1.4 Какими свойствами обладает символ Кронекера}
1. \( \delta_{ij} = \begin{cases} 1 & \text{если } i = j \\ 0 & \text{если } i \neq j \end{cases} \)
2. Симметричность: \( \delta_{ij} = \delta_{ji} \).
3. \( \sum_j \delta_{ij} a_j = a_i \) (свёртка с вектором даёт проекцию).
4. \( \sum_k \delta_{ik} \delta_{kj} = \delta_{ij} \).

\subsection*{1.5 Матрицей линейного оператора \( \varphi \) в базисе \( e_1, e_2 \) некоторого линейного пространства является матрица \( \begin{pmatrix} 1 & 0 \\ -3 & 0 \end{pmatrix} \). Найдите матрицу линейного оператора в базисе \( e'_1 = 2e_1, e'_2 = e_2 \)}
Матрица перехода \( T = \begin{pmatrix} 2 & 0 \\ 0 & 1 \end{pmatrix} \), её обратная \( T^{-1} = \begin{pmatrix} 1/2 & 0 \\ 0 & 1 \end{pmatrix} \).
Новая матрица оператора: \( A' = T^{-1} A T = \begin{pmatrix} 1/2 & 0 \\ 0 & 1 \end{pmatrix} \begin{pmatrix} 1 & 0 \\ -3 & 0 \end{pmatrix} \begin{pmatrix} 2 & 0 \\ 0 & 1 \end{pmatrix} = \begin{pmatrix} 0.5 & 0 \\ -3 & 0 \end{pmatrix} \begin{pmatrix} 2 & 0 \\ 0 & 1 \end{pmatrix} = \begin{pmatrix} 1 & 0 \\ -6 & 0 \end{pmatrix} \).

\subsection*{2.1 Как связаны размерности ядра и образа оператора}
\[\dim_K \ker \varphi + \dim_K \operatorname{Im} \varphi = \dim_K V\]

\subsection*{2.2 Найти собственные значения линейного оператора, матрица которого}
\[\det \left( \begin{pmatrix}
1-\lambda & 2 \\
2 & 1-\lambda
\end{pmatrix} \right) = (1-\lambda)^2 - 2^2 = 0\]
\[(1-\lambda - 2)(1-\lambda + 2) = 0\]
\[(-\lambda - 1)(3-\lambda) = 0\]
\[\lambda = -1 \quad \lambda = 3\]

\subsection*{2.3 Сформулируйте критерии диагонализируемости оператора A}
1. Оператор A диагонализируем тогда и только тогда, когда для каждого его собственного значения λ алгебраическая кратность равна геометрической кратности.
2. Сумма размерностей собственных подпространств равна размерности пространства V (характеристический многочлен раскладывается на линейные множители над полем K, и для каждого собственного значения выполняется условие равенства кратностей).

\subsection*{2.4 Определите алгебраические и геометрические кратности собственных чисел оператора, если в жордановом базисе его матрица имеет вид \( \begin{pmatrix} 1 & 1 & 0 \\ 0 & 1 & 0 \\ 0 & 0 & 2 \end{pmatrix} \)}
Собственное число 1: алгебраическая кратность 2, геометрическая кратность 1 (один жорданов блок размера 2).
Собственное число 2: алгебраическая кратность 1, геометрическая кратность 1.

\subsection*{2.5 Найдите \( e^A \) если \( A = \begin{pmatrix} 1 & 0 & 0 \\ 0 & 0 & 1 \\ 0 & 0 & 0 \end{pmatrix} \)}
Матрица A блочно-диагональная. \( A = \begin{pmatrix} 1 & 0 & 0 \\ 0 & B & 0 \\ 0 & 0 & 0 \end{pmatrix} \), где \( B = \begin{pmatrix} 0 & 1 \\ 0 & 0 \end{pmatrix} \) — нильпотентная матрица 2-го порядка (\(B^2=0\)).
Тогда \( e^A = \begin{pmatrix} e^1 & 0 & 0 \\ 0 & e^B & 0 \\ 0 & 0 & e^0 \end{pmatrix} = \begin{pmatrix} e & 0 & 0 \\ 0 & I + B + \frac{B^2}{2!} + ... & 0 \\ 0 & 0 & 1 \end{pmatrix} = \begin{pmatrix} e & 0 & 0 \\ 0 & \begin{pmatrix} 1 & 1 \\ 0 & 1 \end{pmatrix} & 0 \\ 0 & 0 & 1 \end{pmatrix} = \begin{pmatrix} e & 0 & 0 \\ 0 & 1 & 1 \\ 0 & 0 & 1 \\ 0 & 0 & 0 \end{pmatrix} \). Исправляем размерность: \( e^A = \begin{pmatrix} e & 0 & 0 \\ 0 & 1 & 1 \\ 0 & 0 & 1 \end{pmatrix} \).

\subsection*{3.1 Сформулируйте определение метрического тензора}
Пусть g — билинейная симметричная форма (скалярное произведение). Тогда совокупность чисел \(g_{ij} = g(e_i, e_j)\), вычисленных в базисе \(\{e_1, ..., e_n\}\), называется метрическим тензором.

\subsection*{3.2 Приведите пример скалярного произведения в пространстве квадратных матриц}
\[\langle A, B \rangle = \operatorname{tr}(A^T B)\]

\subsection*{3.3 Какое подпространство называют ортогональным дополнением}
Ортогональным дополнением подпространства L линейного пространства X со скалярным произведением называется множество:
\[L^\perp = \{x \in X : \langle x, y \rangle = 0 \quad \forall y \in L\}\]

\subsection*{3.4 Запишите квадратичную форму по её матрице \( \begin{pmatrix} 1 & 0 & -1 \\ 0 & 1 & 2 \\ -1 & 2 & 0 \end{pmatrix} \)}
\[Q(x) = x^T A x = (x_1, x_2, x_3) \begin{pmatrix} 1 & 0 & -1 \\ 0 & 1 & 2 \\ -1 & 2 & 0 \end{pmatrix} \begin{pmatrix} x_1 \\ x_2 \\ x_3 \end{pmatrix} = x_1^2 + x_2^2 + 0\cdot x_3^2 + 2\cdot(0)x_1x_2 + 2\cdot(-1)x_1x_3 + 2\cdot(2)x_2x_3\]
\[Q(x) = x_1^2 + x_2^2 - 2x_1x_3 + 4x_2x_3\]

\subsection*{3.5 Сформулируйте свойства спектра ортогонального оператора в вещественном евклидовом пространстве}
1. Все собственные значения по модулю равны 1 (\( |\lambda| = 1 \)).
2. Собственные подпространства, отвечающие разным собственным значениям, ортогональны.
3. Оператор может не быть диагонализируемым над \( \mathbb{R} \) (комплексные собственные значения появляются парами \( e^{\pm i\phi} \)), но пространство раскладывается в ортогональную сумму инвариантных подпространств размерности 1 и 2.

\end{document}